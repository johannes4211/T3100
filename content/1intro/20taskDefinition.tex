\Chapter{Aufgabenstellung}
\label{ch:Aufgabenstellung}

Ziel dieser Arbeit ist die Bestimmung der Verlustleistung eines Traktionswechselrichters durch Modellierung und experimentelle Messung.
Zunächst soll für einen einzelnen, definierten Arbeitspunkt ein konsistentes Ergebnis zwischen mathematischer Modellierung und Messung am Antriebsprüfstand erzielt werden. Aufbauend darauf ist die Übertragbarkeit des Ansatzes auf weitere Arbeitspunkte zu untersuchen, um schließlich ein vollständiges Kennfeld des Wechselrichters zu erstellen. Dieses Kennfeld soll es ermöglichen, anhand von Fahrdaten auf den zusätzlichen Energieverbrauch durch Verluste im Inverter zu schließen.

Die Modellierung der Verlustmechanismen erfolgt ausschließlich auf Basis der im Datenblatt des Halbleiterherstellers angegebenen Kennwerte und Charakteristiken. Eine Erweiterung durch zusätzliche empirische Parameter ist nicht vorgesehen.

Die experimentelle Bestimmung der Verluste erfolgt am B2B-Antriebsprüfstand unter Verwendung des vorhandenen Messsystems. Auf den Einsatz eines dedizierten Power Analyzers wird bewusst verzichtet. Stattdessen werden ein- und ausgangsseitige Ströme und Spannungen erfasst und zur Leistungsbestimmung herangezogen.

Die Aufgabenstellung umfasst zudem die Bewertung der erzielbaren Genauigkeit sowie die Identifikation von Grenzen der Modellierung und Messmethodik.

