\Chapter{Aufgabenstellung}
\label{ch:Aufgabenstellung}

\Equation{Boolesche Ungleichung \ref{ch:Aufgabenstellung}}{Boolesche Ungleichung}{
  \begin{equation}
    P(\bigcup_{n=1}^n A_n) \leq \sum_{n=1}^n P(A_n)
  \end{equation}
}

Wenn die Ereignisse $A_n$ disjunkt sind, dann wird die Ungleichung
in \autoref{Boolesche Ungleichung} zu einer Gleichheit. \unsure{Ist das richtig?}
\begin{equation}\label{pythTheorem}
  a^2+b^2=c^2
\end{equation}
Unabhängig davon definiert \autoref{pythTheorem} die Länge der Seiten eines rechtwinkligen Dreiecks, wobei $c$ die Hypotenuse darstellt. \unsure{Und das auch?}



\section{Effizienz in der Formula Student}
In der Formula Student ist die Effizienz des Elektrischen Antriebes ein wichtiger Faktor für die Gesamtbewertung des Fahrzeugs \cite{FSrulebook2026}. Zum einen fließt die verbrauchte Energie während des sog. Endurance Events direkt in die Bewertung ein:
Das Team erhält Punkte basierend auf einem Effizienzfaktor, der sich aus der benötigten Zeit $T$ und der verbrauchten Energie $E$ während des Events berechnet. Der Effizienzfaktor wird als
\begin{equation}
  EF = T^2 \cdot E
\end{equation}
definiert.\\
Punkte werden dann in Abhängigkeit des Effizienzfaktors und des niedrigsten beim Event erreichten Effizienzfaktors vergeben. \cite{FSrulebook2026}\\
\\
Auf der anderen Seite hat die Effizienz des Antriebes auch Einfluss auf die Kühlung der Leistungselektronik und der Elektromotoren. Eine höhere Effizienz bedeutet geringere Verluste und somit eine geringere Abwärme, die abgeführt werden muss. Dies ermöglicht leichtere Kühlsysteme, was wiederum das Gesamtgewicht des Fahrzeugs reduziert und somit die Performance verbessert.

Des Weiteren wird der Hochvolt Akkumulator seitens des DHBW Engineering so ausgelegt, dass er die benötigte Energie für das Endurance Event bereitstellen kann. Ein effizienterer Antrieb benötigt weniger Energie, was die Anforderungen an den Akku reduziert und somit Gewicht und Volumen spart.

