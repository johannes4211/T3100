\Chapter{Technische Grundlagen}


Formula Student Rules --> Warum ist Effizienz wichtig?


n IEC 61800-9-2 definiert den Wirkungsgrad über die Verlustleistung und nicht über die Effizienz selbst im gegensatz zu motoren


The frequency converter international efficiency classes are given in IEC 61800-9-2 [15] as a respect of
reference loss class instead of efficiencies that are used for the electrical machines. The standard uses
the term complete drive module (CDM) to describe all the electronic equipment that are used to control
electrical machines \cite{}

\section{Abschnitt}

Die jeweils ersten Erwähnungen werden immer ausgeschrieben:
\newline
\ac{SiC} und \ac{IGBT}
\newline
Erneut:
\newline
\ac{SiC} und \ac{IGBT}
\subsection{Unterkapitel}

Unterkapitel\ldots



\section{Wirkungsgrad elektrischer Antriebe}
\begin{equation}
    \eta = \dfrac{P_{ein}}{P_{aus}} = \dfrac{P_{ein}-P_{aus}}{P_{ein}} = \dfrac{P_{aus}}{P_{aus}+P_{Verlust}}
\end{equation}




\section{Grundlagen der Leistungsmessung}

\begin{equation}
    P = \dfrac{t_0}{t_0+T} \int_0^T u(t) \cdot i(t) dt
\end{equation}
(Yokogawa Motor Seminar)


\begin{equation}
    P_{m} = \dfrac{2 \cdot \pi \cdot n}{60} \cdot M
\end{equation}



\textbf{Leistungsmessung in 3-Phasigen Systemen}

2- und 3-Wattmeter-Methoden
(Aronschaltung)
--> Was wird bei CSM Modulen genutzt? Anscheinend umstelltbar?!





\section{Messfehler}
 (ZES Zimmer)

% In die Präambel:
% \usepackage{array}  % für >{\raggedright\arraybackslash}
% \usepackage{caption} % optional für bessere Beschriftungen

\begin{table}[htbp]
    \centering
    \caption{Häufig verwendete Methoden zur Bestimmung von Verlusten und Wirkungsgraden nach \cite{GUNDABATTINI2021497}}
    \label{tab:methoden}
    \small
    \begin{tabular}{|>{\raggedright\arraybackslash}p{0.22\textwidth}%
        |>{\raggedright\arraybackslash}p{0.28\textwidth}%
        |>{\raggedright\arraybackslash}p{0.17\textwidth}%
        |>{\centering\arraybackslash}p{0.12\textwidth}%
        |>{\raggedright\arraybackslash}p{0.18\textwidth}|}
        \hline
        \textbf{Methode}                                                  &
        \textbf{Anwendbar für}                                            &
        \textbf{Bedienfreundlichkeit/Komplexität}                         &
        \textbf{Unsicherheit\textsuperscript{3}}                          &
        \textbf{Standardmethode für}                                                                             \\
        \hline
        Input–Output                                                                                             \\‑Methode                                              &
        Alle Geräte (wenn Eingangs‑ und Ausgangsleistung messbar sind)    &
        Einfach und leicht                                                &
        Hoch                                                              &
        DOL‑Motoren\textsuperscript{1}, Frequenzumrichter\textsuperscript{2}, Antriebssysteme\textsuperscript{2} \\
        \hline
        Verlusttrennung                                                   &
        DOL‑Asynchronmotoren (möglicherweise auch DOL‑PMSM und DOL‑SynRM) &
        Komplex                                                           &
        Gering                                                            &
        DOL‑Asynchronmotoren\textsuperscript{1}                                                                  \\
        \hline
        Kalorimetrische Methode                                           &
        Alle Geräte                                                       &
        Äußerst komplex                                                   &
        Sehr gering                                                       &
        Frequenzumrichter\textsuperscript{2}                                                                     \\
        \hline
        Analytische Berechnung                                            &
        Motoren und Frequenzumrichter                                     &
        Komplex                                                           &
        Hoch                                                              &
        Motoren\textsuperscript{1}, Frequenzumrichter\textsuperscript{2}                                         \\
        \hline
        Finite‑Elemente‑Methode                                           &
        Motoren                                                           &
        Komplex                                                           &
        Sehr hoch                                                         &
        --                                                                                                       \\
        \hline
    \end{tabular}

    \vspace{1mm}
    {\footnotesize
        \textsuperscript{1} IEC 60034-2-1 (IEC, 2014b), IEEE 112 (IEEE, 2017).\\
        \textsuperscript{2} IEC 61800-9-2 (IEC, 2017).\\
        \textsuperscript{3} Im Fall hocheffizienter Geräte. Je niedriger die Effizienz des DUT, desto geringer die Unsicherheit der Ein-/Ausgangs‑Methode.
    }
\end{table}




\begin{quote}
    In the frequency converter measurements, the methods used are the input–output method
    and the calorimetric method. These are the only two practical methods for frequency
    converter loss and efficiency determination with tests, and they are consequently defined
    as standard test methods in the standard IEC 61800-9-2
\end{quote} \cite{Karkkainen2021}[p. 53]




Eine klare und physikalisch nachvollziehbare Herleitung dieser Leistungsgrößen unter generalisierten Annahmen lieferte erstmals 1983 ein japanisches Professorenteam um Prof. Hirofumi Akagi, die unter dem Titel „Theory of Instantaneous Real Power and Imaginary Power” oder auch kurz „p-q Theorie“ veröffentlicht wurde.



Maximum



\section{Weiterer Abschnitt}

Beispielhafter Import einer Grafik \change{Das hier ändern}
\begin{figure}[H]
    \centering

    \includegraphics[width=1\textwidth]{media/RED.jpg}

    \captionsetup{width=0.8\textwidth}
    \caption[Kurze Bildbeschreibung]{Ein Bild mit Quellenangabe \cite[S.~9]{Siemens}}

    \label{platzhalterx}
\end{figure}
