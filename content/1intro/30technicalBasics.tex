\Chapter{Technische Grundlagen}
\section{Wechselrichter im elektrischen Antriebsstrang}
Der Antriebsstrang moderner Elektrofahrzeuge besteht in der Regel aus einem Hochvoltakkumulator als Energiespeicher, einem, oder mehreren Wechselstrommotoren, sowie einem oder mehreren Wechselrichtern zur Ansteuerung der Motoren. \\
Der Wechselrichter sorgt dafür, dass der Gleichstrom aus der Batterie in dreiphasigen Wechselstrom für die Motoren umgewandelt wird und besteht aus Halbleiterschaltern, welche mittels \ac{PWM} angesteuert werden. \\
Da Phasenstrom und Spannung im Motor über die Motor- und Drehzahlkonstante direkt mit Drehmoment und Drehzahl verknüpft sind, wird durch die Rückführung der Messung von Phasenströmen und Spannungen und einer Drehmoment- oder Drehzahlvorgabe an den Inverter der Tastgrad $D$ der einzelnen Schalter bestimmt. \\
Die im Wechselrichter enstehenden Verluste werden im Wesentlichen durch die Eigenschaften der Schalter im leitenden, sperrenden und schaltenden Zustand bestimmt.

\section{Wirkungsgrad elektrischer Antriebe}
Der Wirkungsgrad elektrischer Antriebe ist definiert als das Verhältnis aus Eingangs und Ausgangsleistung.
Im betrachteten System ist die Eingangsleistung die von der Batterie kommende, elektrische Gleichstromleistung und die Ausgangsleistung die Mechansiche Leistung des Motors.
\begin{equation}
    \eta = \dfrac{P_{ein}}{P_{aus}} = \dfrac{P_{ein}-P_{aus}}{P_{ein}} = \dfrac{P_{aus}}{P_{aus}+P_{Verlust}}
\end{equation}

Im Allgemeinen liegt der Wirkungsgrad elektrischer Antriebe deutlich über dem eines Fahrzeuges mit Verbrennungsmotor, da die Elektrische direkt in Mechanische Energie umgewandelt werden kann.\\
Die ermittlung des Wirkungsgrades kann entweder durch Messung von Eingnags- und ausgangsleistung, oder durch Messung von Eingangs- oder Ausgangsleistung, sowie der entstandenen Verlus


\begin{table}[h]
    \centering
    \label{tab:verluste_antriebsstrang}
    \begin{tabular}{p{3cm} p{3cm} p{8.5cm}}
        \hline
        \textbf{Komponente}         & \textbf{Verlustart} & \textbf{Beschreibung / Abhängigkeiten}                      \\
        \hline                                                                                                          \\
        Wechselrichter              &
        Leitverluste der Halbleiter &
        Durchlassverluste in Transistoren und Freilaufdioden; abhängig vom Phasenstrom und der Temperatur               \\
        \\
        Wechselrichter              &
        Schaltverluste              &
        Verluste beim Ein- und Ausschalten der Halbleiter; abhängig von Zwischenkreisspannung, Strom und Schaltfrequenz \\
        \\
        Wechselrichter              &
        Nebenverluste               &
        Verluste in Gate-Treibern, Hilfsversorgungen und Steuerung                                                      \\
        \\
        Elektromotor                &
        Kupferverluste              &
        Ohmsche Verluste in den Wicklungen; abhängig vom Statorstrom                                                    \\
        \\
        Elektromotor                &
        Eisenverluste               &
        Hysterese- und Wirbelstromverluste; abhängig von Drehzahl und Magnetfluss                                       \\
        \\
        \hline
    \end{tabular}
    \caption{Übersicht über Verlustmechanismen im elektrischen Antriebsstrang}
\end{table}



\subsection{Grundlagen der Leistungsmessung}

\begin{equation}
    P = \dfrac{t_0}{t_0+T} \int_0^T u(t) \cdot i(t) dt
\end{equation}

\begin{equation}
    P_{m} = \dfrac{2 \cdot \pi \cdot n}{60} \cdot M
\end{equation}



\subsection{3-Wattmeter Methode}
\unsure{add text}

\begin{figure}
    \centering
    \includegraphics[scale = 0.4]{media/drawio/3P3W.drawio.png}
    \caption{Aufbau der 3-Wattmeter Messung}
    \label{fig: WP3W}
\end{figure}

\begin{gather}
    P_{ges} = u_R \cdot i_R + u_S \cdot i_S + u_T \cdot i_T\\
    P_1 = i_R (u_R - u_T) = i_R \cdot u_{RT}\\
    P_2 = i_S (u_S - u_T) = i_S \cdot u_{ST}\\
    i_R + i_S + i_T = 0 \\
    i_T = -(i_R + i_S) \\
    P_{ges} = u_R \cdot i_R + u_S \cdot i_S + u_T \cdot (-(i_R + i_S)) \\
    i_R \cdot (u_R - u_T) + i_S \cdot (u_S - u_T)
\end{gather}

Die Stern-Dreieck-Transformation ist notwendig, um die Spannung der einzelnen Phasen zum Sternpunkt zu erhalten.









%\section{Messfehler}

% In die Präambel:
% \usepackage{array}  % für >{\raggedright\arraybackslash}
% \usepackage{caption} % optional für bessere Beschriftungen

\begin{table}[htbp]
    \centering
    \caption{Häufig verwendete Methoden zur Bestimmung von Verlusten und Wirkungsgraden nach \cite{GUNDABATTINI2021497}}
    \label{tab:methoden}
    \small
    \begin{tabular}{|>{\raggedright\arraybackslash}p{0.22\textwidth}%
        |>{\raggedright\arraybackslash}p{0.28\textwidth}%
        |>{\raggedright\arraybackslash}p{0.17\textwidth}%
        |>{\centering\arraybackslash}p{0.12\textwidth}%
        |>{\raggedright\arraybackslash}p{0.18\textwidth}|}
        \hline
        \textbf{Methode}                                                  &
        \textbf{Anwendbar für}                                            &
        \textbf{Bedienfreundlichkeit/Komplexität}                         &
        \textbf{Unsicherheit\textsuperscript{3}}                          &
        \textbf{Standardmethode für}                                                                             \\
        \hline
        Input–Output                                                                                             \\‑Methode                                              &
        Alle Geräte (wenn Eingangs‑ und Ausgangsleistung messbar sind)    &
        Einfach und leicht                                                &
        Hoch                                                              &
        DOL‑Motoren\textsuperscript{1}, Frequenzumrichter\textsuperscript{2}, Antriebssysteme\textsuperscript{2} \\
        \hline
        Verlusttrennung                                                   &
        DOL‑Asynchronmotoren (möglicherweise auch DOL‑PMSM und DOL‑SynRM) &
        Komplex                                                           &
        Gering                                                            &
        DOL‑Asynchronmotoren\textsuperscript{1}                                                                  \\
        \hline
        Kalorimetrische Methode                                           &
        Alle Geräte                                                       &
        Äußerst komplex                                                   &
        Sehr gering                                                       &
        Frequenzumrichter\textsuperscript{2}                                                                     \\
        \hline
        Analytische Berechnung                                            &
        Motoren und Frequenzumrichter                                     &
        Komplex                                                           &
        Hoch                                                              &
        Motoren\textsuperscript{1}, Frequenzumrichter\textsuperscript{2}                                         \\
        \hline
        Finite‑Elemente‑Methode                                           &
        Motoren                                                           &
        Komplex                                                           &
        Sehr hoch                                                         &
        --                                                                                                       \\
        \hline
    \end{tabular}

    \vspace{1mm}
    {\footnotesize
        \textsuperscript{1} IEC 60034-2-1 (IEC, 2014b), IEEE 112 (IEEE, 2017).\\
        \textsuperscript{2} IEC 61800-9-2 (IEC, 2017).\\
        \textsuperscript{3} Im Fall hocheffizienter Geräte. Je niedriger die Effizienz des DUT, desto geringer die Unsicherheit der Ein-/Ausgangs‑Methode.
    }
\end{table}



\subsection{Kaloriemetrische Messung}
Die Kaloriemetrische Messung sollte in einer abgeschlossenen Kammer stattfinden und könnte für den Wecselrichter des DHBWE z.B. über messung des Temepraturanstieges des Kühlwassers erfolgen, wird aber aufgrund der hohen ungenauigkeit im folgenden nicht berücksichtigt.
\begin{equation}
    \Delta T_{Wasser} = \frac{\dot{Q}}{\dot{m} \cdot c_p}
\end{equation}









