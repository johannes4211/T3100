\Chapter{Technische Grundlagen}
IEC 61800-9-2 definiert den Wirkungsgrad über die Verlustleistung und nicht über die Effizienz selbst im gegensatz zu motoren

\section{Wirkungsgrad elektrischer Antriebe}
\begin{equation}
    \eta = \dfrac{P_{ein}}{P_{aus}} = \dfrac{P_{ein}-P_{aus}}{P_{ein}} = \dfrac{P_{aus}}{P_{aus}+P_{Verlust}}
\end{equation}

\section{Grundlagen der Leistungsmessung}

\begin{equation}
    P = \dfrac{t_0}{t_0+T} \int_0^T u(t) \cdot i(t) dt
\end{equation}

\begin{equation}
    P_{m} = \dfrac{2 \cdot \pi \cdot n}{60} \cdot M
\end{equation}



\subsection{3-Wattmeter Methode}

\begin{gather}
    P_{ges} = u_R \cdot i_R + u_S \cdot i_S + u_T \cdot i_T\\
    P_1 = i_R (u_R - u_T) = i_R \cdot u_{RT}\\
    P_2 = i_S (u_S - u_T) = i_S \cdot u_{ST}\\
    i_R + i_S + i_T = 0 \\
    i_T = -(i_R + i_S) \\
    P_{ges} = u_R \cdot i_R + u_S \cdot i_S + u_T \cdot (-(i_R + i_S)) \\
    i_R \cdot (u_R - u_T) + i_S \cdot (u_S - u_T)
\end{gather}

Die Stern-Dreieck-Transformation ist notwendig, um die Spannung der einzelnen Phasen zum Sternpunkt zu erhalten.

\begin{figure}
    \centering
    \includegraphics{media/3P3W.png}
    \caption{3- Wa}
\end{figure}

\textbf{Aronschaltung}






%\section{Messfehler}

% In die Präambel:
% \usepackage{array}  % für >{\raggedright\arraybackslash}
% \usepackage{caption} % optional für bessere Beschriftungen

\begin{table}[htbp]
    \centering
    \caption{Häufig verwendete Methoden zur Bestimmung von Verlusten und Wirkungsgraden nach \cite{GUNDABATTINI2021497}}
    \label{tab:methoden}
    \small
    \begin{tabular}{|>{\raggedright\arraybackslash}p{0.22\textwidth}%
        |>{\raggedright\arraybackslash}p{0.28\textwidth}%
        |>{\raggedright\arraybackslash}p{0.17\textwidth}%
        |>{\centering\arraybackslash}p{0.12\textwidth}%
        |>{\raggedright\arraybackslash}p{0.18\textwidth}|}
        \hline
        \textbf{Methode}                                                  &
        \textbf{Anwendbar für}                                            &
        \textbf{Bedienfreundlichkeit/Komplexität}                         &
        \textbf{Unsicherheit\textsuperscript{3}}                          &
        \textbf{Standardmethode für}                                                                             \\
        \hline
        Input–Output                                                                                             \\‑Methode                                              &
        Alle Geräte (wenn Eingangs‑ und Ausgangsleistung messbar sind)    &
        Einfach und leicht                                                &
        Hoch                                                              &
        DOL‑Motoren\textsuperscript{1}, Frequenzumrichter\textsuperscript{2}, Antriebssysteme\textsuperscript{2} \\
        \hline
        Verlusttrennung                                                   &
        DOL‑Asynchronmotoren (möglicherweise auch DOL‑PMSM und DOL‑SynRM) &
        Komplex                                                           &
        Gering                                                            &
        DOL‑Asynchronmotoren\textsuperscript{1}                                                                  \\
        \hline
        Kalorimetrische Methode                                           &
        Alle Geräte                                                       &
        Äußerst komplex                                                   &
        Sehr gering                                                       &
        Frequenzumrichter\textsuperscript{2}                                                                     \\
        \hline
        Analytische Berechnung                                            &
        Motoren und Frequenzumrichter                                     &
        Komplex                                                           &
        Hoch                                                              &
        Motoren\textsuperscript{1}, Frequenzumrichter\textsuperscript{2}                                         \\
        \hline
        Finite‑Elemente‑Methode                                           &
        Motoren                                                           &
        Komplex                                                           &
        Sehr hoch                                                         &
        --                                                                                                       \\
        \hline
    \end{tabular}

    \vspace{1mm}
    {\footnotesize
        \textsuperscript{1} IEC 60034-2-1 (IEC, 2014b), IEEE 112 (IEEE, 2017).\\
        \textsuperscript{2} IEC 61800-9-2 (IEC, 2017).\\
        \textsuperscript{3} Im Fall hocheffizienter Geräte. Je niedriger die Effizienz des DUT, desto geringer die Unsicherheit der Ein-/Ausgangs‑Methode.
    }
\end{table}



\subsection{Kaloriemetrische Messung}
Die Kaloriemetrische Messung sollte in einer abgeschlossenen Kammer stattfinden und könnte für den Wecselrichter des DHBWE z.B. über messung des Temepraturanstieges des Kühlwassers erfolgen, wird aber aufgrund der hohen ungenauigkeit im folgenden nicht berücksichtigt.
\begin{equation}
    \Delta T_{Wasser} = \frac{\dot{Q}}{\dot{m} \cdot c_p}
\end{equation}









