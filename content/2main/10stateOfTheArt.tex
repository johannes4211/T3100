\Chapter{Stand der Technik}

Einführung\ldots

Dies ist die Einführung in das Kapitel Stand der Technik.
Mit vielen interessanten Informationen !

\section{Halbleitermodule}
Die Leistungselektronik in modernen Antriebssträngen basiert auf Halbleitermodulen, die verschiedene Halbleitermaterialien und Topologien verwenden, um elektrische Energie effizient zu steuern und umzuwandeln.
Zu den gängigen Halbleitermaterialien gehören Silizium (Si), Siliziumkarbid (SiC) und Galliumnitrid (GaN).
Diese Materialien bieten unterschiedliche Vorteile hinsichtlich Schaltgeschwindigkeit, Wärmeleitfähigkeit und Spannungsfestigkeit.
Moderne Halbleitermodule sind in verschiedenen Bauformen erhältlich, darunter diskrete Bauelemente, Module mit mehreren Halbleitern und integrierte Systeme.
Die Wahl der Topologie, wie z.B. B6-Brücke, NPC (Neutral Point Clamped) oder T-Type, hängt von den spezifischen Anforderungen der Anwendung ab, einschließlich Effizienz, Leistungsdichte und Kosten.
Fortschritte in der Halbleitertechnologie haben zu einer verbesserten Leistung und Zuverlässigkeit von Wechselrichtern geführt, was wiederum die Entwicklung von Elektrofahrzeugen und erneuerbaren Energiesystemen vorantreibt.

IGBT, Si, SiC, GaN, V6

Bauformen, Topologien


Wichtige Kennzahlen eines Datenblattes --> eher bei technische Grundlagen
Rds on, Eon, Eoff



pq Methode zur Mssung der instantanmem Leistung in 3-Phasigen Systemen


\section{Messysteme}
Die Leistungsmessung zur Bestimmung des Wirkungsgrades für Industrielle Anwendungen mit höchster Genauigkeit erfolgt in der Regel mit sog. Leistungsanalysatoren.
%https://cdn.tmi.yokogawa.com/11/3823/details/Motor_Seminar_MTG.pdf


\cite{Spannungsmessung}
\cite{Strommessung}



\section{Antriebsstrang DHBW Engineering}
Der Antriebsstrang des eSleek26 besteht aus vier Radnabenmaschinen, die jeweils von einem eigenen Wechselrichter angesteuert werden.
Die vier Wechselrichter sind zentral im Fahrzeug verbaut und werden durch einen 600 V Hochvolt Akkumulator gespeist.

\subsection{Traktionswechselrichter PrePulse}
Seit der Saison 2023 setzt das DHBW Engineering auf einen Eigenentwickelten Traktionswechselrichter.
Dieser basiert auch auf SiC Halbleitermodulen in B6 Topder Fa. Microchip.
Die Module bestehen aus 3 Phasenbrücken mit jeweils 2 SiC MOSFETs und 2 SiC Dioden pro Halbbrücke, welche




\begin{figure}
    \center
    \includegraphics[scale = 0.07]{media/B6_upscaled.png}
    \caption{B6 Modul des Wechselrichters nach \cite{Microchip}}
    \label{fig: Wechselrichter Aufbau}
\end{figure}


\subsection{Elektrische Maschine}



\subsection{Antriebsprüfstand}
Der Antriebsprüfstand des DHBWE aus zwei elektrischen Maschinen, welche mechanisch über eine Drehmomentmesswelle verbunden sind.
Auf der einen Seite der Welle befindet sich der Testmotor, auf der anderen eine ebenfalls Permanenterregte Lastmaschine vom Typ AMK DD7 \unsure{hier genaue Bezeichnung einfügen u. Datenblatt verlinken}.
Der Testmotor wird durch den eigenentwickelten Wechselrichter PrePulse \unsure{wie als eigennamen Markieren?} angesteuert, während die Lastmaschine über einen Wechselrichter der Fa. AMK vom Typ KW60\unsure{Datenblatt!} angesteuert wird. \\
Im Betrieb zur Vermessung von Prüflingsmotor und -inverter wird die Lastmaschine Drezahlgeregelt angesteuert und der Prüflingsmotor Momentengeregelt.
D.h., dass zunächst eine Solldrehzahl an den Lastinverter geschickt wird, welcher solange ein positives Drehmoment am Motor stellt, bis die Solldrehzahl erreicht ist.\\
Anschließend wird ein Sollmoment an den Prüflingsinverter geschickt, welcher dafür sorgt, dass der zur Erzeugung Drehmoments nötige Strom im Motor fließt.

AUF PTP Synchronisation eingehen!


AC Leistungsberechnung: ist in den Messdaten eine Phasenverschiebung zwischen Strom und Spannung enthalten --> Leistungsfaktor anschauen \unsure{Synchronisierungsproblem mit PTP?}



\subsubsection{CSM Messmodule}
Messung über Shunt und HW-Filter mit Grenzfrequenz 250 kHz


3x CSM HV BM 1.2 zur AC Messung
1x CSM HV BM 1.1 zur DC Messung
2x CSM XCP Gateways zur Kommunikation zum MessPC
1x AD-Scan MiniModul zur Drehmomentbestimmung
1x CNT4 evo zur Drehzahlbestimmung
1x VP6450 als Schnittstelle zwischen den Gateways/CNT4/AD-Scan und dem PrüfstandsPC
%https://docs.google.com/document/d/%1Xq_vhz7K0Xn5hN8SluS7x-uVCBVajVXi1cco3dHOIgA/edit?tab=t.0



\begin{figure}
    \center
    \includegraphics[scale = 0.2]{media/drawio/B2B_Pruefstand.drawio.png}
    \caption{Back-to-Back Prüfstand mit Messystem}
    \label{fig: B2B}
\end{figure}



Kaloriemetrische Messung
\begin{equation}
    \Delta T_{Wasser} = \frac{\dot{Q}}{\dot{m} \cdot c_p}
\end{equation}



Rückrechnung mit Motorverlusten








SVPWM --> in keinem der 6 Bereiche muss der Zeiger durch das positiv schalten von 6 Schaltern laufen



\section{Messunsicherheit}

\cite{Karkkainen2021}


