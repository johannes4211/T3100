\Chapter{Stand der Technik}

\section{Halbleitermodule}
Die Leistungselektronik in modernen Antriebssträngen basiert auf Halbleitermodulen, die verschiedene Halbleitermaterialien und Topologien verwenden, um elektrische Energie effizient zu steuern und umzuwandeln.
Zu den gängigen Halbleitermaterialien gehören Silizium (Si), Siliziumkarbid (SiC) und Galliumnitrid (GaN).
Diese Materialien bieten unterschiedliche Vorteile hinsichtlich Schaltgeschwindigkeit, Wärmeleitfähigkeit und Spannungsfestigkeit.
Moderne Halbleitermodule sind in verschiedenen Bauformen erhältlich, darunter diskrete Bauelemente, Module mit mehreren Halbleitern und integrierte Systeme.
Die Wahl der Topologie, wie z.B. B6-Brücke, NPC (Neutral Point Clamped) oder T-Type, hängt von den spezifischen Anforderungen der Anwendung ab, einschließlich Effizienz, Leistungsdichte und Kosten.
Fortschritte in der Halbleitertechnologie haben zu einer verbesserten Leistung und Zuverlässigkeit von Wechselrichtern geführt, was wiederum die Entwicklung von Elektrofahrzeugen und erneuerbaren Energiesystemen vorantreibt.


\section{Messysteme}
Die Leistungsmessung zur Bestimmung des Wirkungsgrades für Industrielle Anwendungen mit höchster Genauigkeit erfolgt in der Regel mit sog. Leistungsanalysatoren.
%https://cdn.tmi.yokogawa.com/11/3823/details/Motor_Seminar_MTG.pdf



\section{Antriebsstrang DHBW Engineering}
Der Antriebsstrang des eSleek26 besteht aus vier Radnabenmaschinen, die jeweils von einem eigenen Wechselrichter angesteuert werden.
Die vier Wechselrichter sind zentral im Fahrzeug verbaut und werden durch einen 600 V Hochvolt Akkumulator gespeist.

\subsection{Traktionswechselrichter PrePulse}
Seit der Saison 2023 setzt das DHBW Engineering auf einen Eigenentwickelten Traktionswechselrichter.
Dieser basiert auch auf SiC Halbleitermodulen in B6 Topder Fa. Microchip.
Die Module bestehen aus 3 Phasenbrücken mit jeweils 2 SiC MOSFETs und 2 SiC Dioden pro Halbbrücke, welche




\begin{figure}
    \center
    \includegraphics[scale = 0.07]{media/B6_upscaled.png}
    \caption{B6 Modul des Wechselrichters nach \cite{Microchip}}
    \label{fig: Wechselrichter Aufbau}
\end{figure}


\subsection{Antriebsprüfstand}
Der Antriebsprüfstand des DHBWE aus zwei elektrischen Maschinen, welche mechanisch über eine Drehmomentmesswelle verbunden sind.
Auf der einen Seite der Welle befindet sich der Testmotor, auf der anderen eine ebenfalls Permanenterregte Lastmaschine vom Typ AMK DD7.
Der Testmotor wird durch den eigenentwickelten Wechselrichter PrePulse angesteuert, während die Lastmaschine über einen Wechselrichter der Fa. AMK vom Typ KW60 angesteuert wird. \\
Im Betrieb zur Vermessung von Prüflingsmotor und -inverter wird die Lastmaschine Drezahlgeregelt angesteuert und der Prüflingsmotor Momentengeregelt.
D.h., dass zunächst eine Solldrehzahl an den Lastinverter geschickt wird, welcher solange ein positives Drehmoment am Motor stellt, bis die Solldrehzahl erreicht ist.\\
Anschließend wird ein Sollmoment an den Prüflingsinverter geschickt, welcher dafür sorgt, dass der zur Erzeugung Drehmoments nötige Strom im Motor fließt.

\subsubsection{CSM Messmodule}
Die Messung von Strom und Spannung erfolg am Motorprüfstand des DHBW Engineering mittels HV-Breakout Modulen der Fa. Computer Systeme Messtechnik GmbH. \\
Dabei werden zwei Module der Version 1.2 \cite{BM1.2} zur AC Strom und Spannungsmessung nach der \ac{3W3P} Methode verwendt
Messung über Shunt und HW-Filter mit Grenzfrequenz 250 kHz




\begin{figure}
    \center
    \includegraphics[scale = 0.1]{media/drawio/B2B_Pruefstand.drawio.png}
    \caption{Back-to-Back Prüfstand mit Messystem}
    \label{fig: B2B}
\end{figure}





