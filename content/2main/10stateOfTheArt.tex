\Chapter{Stand der Technik}

Einführung\ldots

Dies ist die Einführung in das Kapitel Stand der Technik.
Mit vielen interessanten Informationen !

\section{Halbleitermodule}
IGBT, Si, SiC, GaN, V6

Bauformen, Topologien


Wichtige Kennzahlen eines Datenblattes --> eher bei technische Grundlagen
Rds on, Eon, Eoff


\section{Messysteme}

\cite{Spannungsmessung}
\cite{Strommessung}

MotorPrüfstand DHBW Engineering --> Messaufbau zeigen


AC Leistungsberechnung: ist in den Messdaten eine Phasenverschiebung zwischen Strom und Spannung enthalten --> Leistungsfaktor anschauen \unsure{Synchronisierungsproblem mit PTP?}



Leistungsanalysatoren
%https://cdn.tmi.yokogawa.com/11/3823/details/Motor_Seminar_MTG.pdf





Kaloriemetrische Messung
\begin{equation}
    \Delta T_{Wasser} = \frac{\dot{Q}}{\dot{m} \cdot c_p}
\end{equation}



Rückrechnung mit Motorverlusten








SVPWM --> in keinem der 6 Bereiche muss der Zeiger durch das positiv schalten von 6 Schaltern laufen



\section{Messunsicherheit}

\cite{Karkkainen2021}