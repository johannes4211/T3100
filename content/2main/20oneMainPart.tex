\Chapter[Spezifikation Kurztitel]{Erwartete Verluste}

Einführung im \autoref{Spezifikation Kurztitel}\ldots



\section{Validierung des Datenblattes durch Doppelpulstest}

Die verschiedenen Kennlinien Werden mit den ergebnisssen eines im Jahr 2023 durchgeführen Doppelpulstests \ac{DPT} verglichen.

\cite{DPT_Luca}



\section{Erwartete Verluste}

\begin{Itemize}
  \item Formulierung quantitativer Leistungskennzahlen
  \newline
  Definition quantitativer KPIs ist von Bedeutung, um die Leistungsfähigkeit der
  verschiedenen Simulationssystem-Architekturen objektiv zu bewerten. Dieses Teilziel
  beinhaltet die Identifizierung geeigneter KPIs und die Entwicklung von Metriken zur Messung
  der Simulationskapazität
  \item Aufbau des Testmodells
  \newline
  Die Entwicklung des konfigurierbaren  dsTestmodells bildet die Grundlage für die Durchführung von
  Tests an verschiedenen Simulationssystem-Architekturen. Hierbei gilt es, die Plattform so zu gestalten,
  dass sie flexibel an verschiedene Simulationsszenarien anpassbar ist
  \item Entwicklung eines flexiblen Konfigurationssystems \improvement{This really needs to be improved! What was I thinking?!}
  \newline
  Die Implementierung eines flexiblen Konfigurationssystems ermöglicht die Anpassung des Testmodells an
  unterschiedliche Simulationskapazitäten. Hierbei müssen Parameter definiert werden, die es dem Testmodell
  ermöglichen, den Simulationsaufwand je nach Architektur anzupassen
\end{Itemize}



\begin{equation}
  P_{sw,bridge} \approx 6 \cdot f_{sw} \cdot k_{SVPWM} \cdot \frac{1}{2\pi} \int_0^{2\pi} \left(E_{on}(|i(t)|) + E_{off}(|i(t)|)\right) \, d(\omega t)
\end{equation}





Messung über Shunt und HW-Filter mit Grenzfrequenz 250 kHz.
3x CSM HV BM 1.2 zur AC Messung
1x CSM HV BM 1.1 zur DC Messung
2x CSM XCP Gateways zur Kommunikation zum MessPC
1x AD-Scan MiniModul zur Drehmomentbestimmung
1x CNT4 evo zur Drehzahlbestimmung
1x VP6450 als Schnittstelle zwischen den Gateways/CNT4/AD-Scan und dem PrüfstandsPC
