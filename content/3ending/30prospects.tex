\Chapter{Ausblick}

Aufbauend auf den Ergebnissen der Arbeit gilt es nun die verbliebene Messunsichereheit durch Anpassung der zeitlichen Synchronisation auszuräumen.
Weitere mögliche Fehlerursachen der Messung sind noch zu untersuchen.

Für eine genauere Modellierung der im Halbleiter entstehenden Verluste muss die berechnung um eine Temperaturabhängigkeit erweitert werden.
Dazu sollten zunächst die im Datenblatt vorandenen Angaben zum $R_{DS_{on}}$
verwendet werden, um die Verlauf der Leitverluste besser zu modellieren.

Kann durch diese Maßnahmen eine bessere Übereinstimmung der Modellierung und der Messung am Prüfstand erreicht werden, ist die Messung auf weitere relevante Arbeitpunkte auszuweiten und ein vollständiges Effizienzkennfeld zu ermitteln.