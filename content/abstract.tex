\renewcommand{\abstractname}{\LARGE Kurzreferat}
\begin{abstract}
  \vspace{1.2cm}
  In der Formula Student sowie der Elektromobilität im Allgemeinen spielt Energieeffizienz eine wichtige Rolle.\\
  Um den Antriebsstrang zu optimieren, ist eine genaue Kenntnis der auftretenden Verlustleistung der Einzelkomponenten notwendig. \\
  Die vorliegende Arbeit beschäftigt sich mit der Bestimmung der Verlustleistung eines Traktionswechselrichters auf der einen Seite durch Modellierung der Verlustmechanismen und auf der anderen Seite messtechnisch mit der Bestimmung des Gesamtwirkungsgrades in Abhängigkeit des Betriebszustandes.
  Ziel ist es, ein konsistentes Ergebnis zwischen einfacher mathematischer Modellierung und der Vermessung am Antriebsprüfstand für jeden im Fahrzeug auftretenden Arbeitspunkt zu bestimmen, um auf den zusätzlichen Energieverbrauch durch Verluste im Inverter zurückschließen zu können.
  Die Modellierung erfolgt mithilfe der Kennlinien aus dem Datenblatt des Halbleiterherstellers. \\
  Die Messung am Antriebsprüfstand erfolgt über die Messung von ein- und ausgangsseitigen Strömen und Spannungen sowie den Vergleich der Leistungen. \\
  Aufgrund vereinfachter Verlustmodelle und messtechnischer Unsicherheiten, insbesondere bei der zeitlichen Synchronisation und Auflösung der Messwerte, zeigen die modellierten und experimentell ermittelten Ergebnisse derzeit keine ausreichende Übereinstimmung.


  \section*{Schlüsselwörter}
  Elektromobilität \kwtextbullet Antriebsstrang \kwtextbullet Inverter \kwtextbullet Effizienz \kwtextbullet Wirkungsgrad \kwtextbullet Prüfstand \kwtextbullet Leistungsmessung

\end{abstract}

% LTeX: enabled=true language=en-US

\renewcommand{\abstractname}{\LARGE Abstract}
\begin{abstract}
  \vspace{1.2cm}
  In Formula Student as well as in electromobility in general, energy efficiency plays an important role.\\
  To optimize the powertrain, precise knowledge of the power losses occurring in the individual components is required. \\
  This rport focuses on the determination of the power losses of a traction inverter, on the one hand by modeling the loss mechanisms and on the other hand experimentally by determining the overall efficiency for relevant operating points.
  The objective is to obtain consistent results between a simplified mathematical model and measurements conducted on a powertrain test bench for each operating point occurring in the vehicle, in order to assess the additional energy consumption caused by losses in the inverter.
  The modeling is carried out using characteristic curves provided in the semiconductor manufacturer’s datasheet. \\
  The measurements on the powertrain test bench are performed by measuring input- and output-side currents and voltages and comparing the corresponding power levels. \\
  Due to simplified loss models and measurement uncertainties, particularly with regard to the temporal synchronization and resolution of the measurement data, the modeled and experimentally obtained results currently do not show sufficient agreement.


  \section*{Keywords}
  Electromobility \kwtextbullet Powertrain \kwtextbullet Inverter \kwtextbullet Efficiency \kwtextbullet Efficiency factor \kwtextbullet Test bench \kwtextbullet Power measurement

\end{abstract}
